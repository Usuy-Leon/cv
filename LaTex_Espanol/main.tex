\documentclass[10pt, letterpaper]{article}

% Packages:
\usepackage[
  top=2 cm,
  bottom=2 cm,
  left=2 cm,
  right=2 cm,
  footskip=1.0 cm
]{geometry}

\usepackage{tikz}
\usepackage{titlesec}
\usepackage{tabularx}
\usepackage{array}
\usepackage[dvipsnames]{xcolor}
\usepackage[T1]{fontenc}
\usepackage{lato}

\definecolor{primaryColor}{RGB}{0, 0, 0}

\usepackage{enumitem}
\usepackage{amsmath}

\usepackage[
  colorlinks=true,
  urlcolor=primaryColor
]{hyperref}

\usepackage[pscoord]{eso-pic}
\usepackage{calc}
\usepackage{bookmark}
\usepackage{lastpage}
\usepackage{changepage}
\usepackage{paracol}
\usepackage{ifthen}
\usepackage{needspace}
\usepackage{iftex}
\usepackage{multicol}
\usepackage{tikz}
\usetikzlibrary{calc}

\usepackage{etoolbox} % for \AtBeginEnvironment

% Font encoding
\ifPDFTeX
  \input{glyphtounicode}
  \pdfgentounicode=1
  \usepackage[T1]{fontenc}
  \usepackage[utf8]{inputenc}
  \usepackage{lmodern}
\fi

\usepackage{charter}

% Formatting
\raggedright
\AtBeginEnvironment{adjustwidth}{\partopsep=0pt}
\pagestyle{empty}
\setcounter{secnumdepth}{0}
\setlength{\parindent}{0pt}
\setlength{\topskip}{0pt}
\setlength{\columnsep}{0.15cm}
\pagenumbering{gobble}

\titleformat{\section}{\needspace{4\baselineskip}\bfseries\large}{}{0pt}{}[\vspace{1pt}\titlerule]
\titlespacing{\section}{-1pt}{0.4 cm}{0.2 cm}

\renewcommand\labelitemi{$\vcenter{\hbox{\small$\bullet$}}$}

% Highlights environment
\newenvironment{highlights}{
  \begin{itemize}[
    topsep=0.10 cm,
    parsep=0.10 cm,
    partopsep=0pt,
    itemsep=0pt,
    leftmargin=10pt
  ]
}{
  \end{itemize}
}

\newenvironment{highlightsforbulletentries}{
  \begin{itemize}[
    topsep=0.10 cm,
    parsep=0.10 cm,
    partopsep=0pt,
    itemsep=0pt,
    leftmargin=10pt
  ]
}{
  \end{itemize}
}

% Single column wrapper
\newenvironment{onecolentry}{
  \begin{adjustwidth}{0.00001cm}{0.00001cm}
}{
  \end{adjustwidth}
}

% --- Connected bullets (stable version) ---
\usepackage{tikz}
\usetikzlibrary{calc}

\newcounter{cbitem}

% Environment
\newenvironment{connectedbullets}{
  \setcounter{cbitem}{0}%
  \begin{itemize}[leftmargin=1.6cm, itemsep=1.2em]
}{
  \end{itemize}
}

% Bullet + connecting line
\newcommand{\connecteditem}{%
    \stepcounter{cbitem}%
    \edef\cbcurrent{cb\thecbitem}%

    % Draw the bullet node
    \item[\tikz[remember picture,baseline=-0.6ex]{%
        \node (\cbcurrent) [circle, fill=forestgreen,
            minimum size=5pt, inner sep=0pt] {};
    }]%

    % Draw the connector *after* the item, ensuring node exists
    \ifnum\value{cbitem}>1
        \tikz[remember picture,overlay]{%
            \draw[forestgreen,line width=0.6pt]
                ($(cb\the\numexpr\value{cbitem}-1\relax.south)$)
                -- ($(cb\thecbitem.north)$);
        };
    \fi
}


% Two-column environment (fixed)
\newenvironment{twocolentry}[1]{%
  \def\RightColumn{#1}%
  \begin{onecolentry}%
  \begin{paracol}{2}%
  \setlength{\columnsep}{0.15cm}%
  \raggedright%
}{%
  \switchcolumn%
  \raggedleft \RightColumn%
  \end{paracol}%
  \end{onecolentry}%
}

% Header environment
\newenvironment{header}{
  \setlength{\topsep}{0pt}\par\kern\topsep\centering\linespread{1.5}
}{
  \par\kern\topsep
}

% "Last updated" marker
\newcommand{\placelastupdatedtext}{%
  \AddToShipoutPictureFG*{%
    \put(\LenToUnit{\paperwidth-0.3cm}, \LenToUnit{\paperheight-0 cm}){%
      \vtop{{\null}\makebox[0pt][c]{%
        \small\color{white}\textit{Actualizado Octubre 2025}\hspace{\widthof{Actualizado Octubre 2025}}
      }}%
    }%
  }%
}

\let\hrefWithoutArrow\href

% Dark forest green color (#145A14)
\definecolor{forestgreen}{RGB}{20,90,20}
\definecolor{olivegreen}{RGB}{20, 90, 20}
\begin{document}
% This will overlay the house icon as a clickable link in the upper-left corner of the page
\begin{tikzpicture}[remember picture,overlay]
  \node[anchor=north west, xshift=10pt, yshift=-10pt] at (current page.north west) 
  {\href{https://usuy-leon.github.io/cv/}{\includegraphics[width=48pt]{home.png}}};
\end{tikzpicture}

\placelastupdatedtext

\begin{tikzpicture}[remember picture, overlay]
  % Rectangle covering top 8 cm
  \fill[forestgreen] (current page.north west) rectangle ([yshift=-8cm]current page.north east);

  % Title centered
  \node[
    white,
    font=\fontsize{25pt}{25pt}\selectfont,
    align=center,
    anchor=north,
    text width=\paperwidth,
    yshift=-1.5cm
  ] at (current page.north) {Usuy David León Tolosa};

  % Contact info (corrected hyperlinks)
  \node[
    white,
    font=\normalsize,
    align=center,
    anchor=north,
    text width=\paperwidth,
    yshift=-3.1cm
  ] at (current page.north) {%
    \sffamily
    \kern 5pt \textbar \kern 5pt Colombia \kern 5pt \textbar \kern 5pt
    \href{mailto:udleont@unal.edu.co}{\textcolor{white}{udleont@unal.edu.co}} \kern 5pt \textbar \kern 5pt
    \href{https://orcid.org/0000-0001-7868-9934}{\textcolor{white}{ORCID}} \kern 5pt \textbar \kern 5pt
    \href{https://github.com/Usuy-Leon}{\textcolor{white}{GitHub}} \kern 5pt \textbar \kern 5pt
    \href{https://usuy-leon.github.io/cv//}{\textcolor{white}{Web}}
  };

  % Section title
  \node[
    white,
    font=\bfseries\large,
    align=center,
    anchor=north,
    text width=\paperwidth,
    yshift=-4.1cm
  ] at (current page.north) {Resumen};

  % Resumen text
  \node[
    white,
    font=\normalsize,
    align=justify,
    anchor=north,
    text width=0.85\paperwidth,
    yshift=-5.0cm
  ] at (current page.north)
  {
    Investigador y educador con 2 años de experiencia después de explorar los mecanismos biológicos que subyacen a la formación de tejidos, la retina y la visión. Mi propósito profesional es aportar activamente a la construcción de una Colombia en paz, que permita soñar con el cuidado de la vida, la ciencia y la innovación. Busco impulsar procesos colectivos, de diversidad, educación y de ciencia abierta, que nos permitan superar el legado de violencia que de una u otra forma todos los colombianos llevamos.
  };

\end{tikzpicture}

    \vspace*{5.3cm} % Space after the fixed block

        \textcolor{forestgreen}{\section{Formación académica}}

    \begin{twocolentry} {\textcolor{forestgreen}{Enero 2016 -- Diciembre 2022}}
        \textcolor{olivegreen}{\textbf{Biólogo}}: \textbf{Universidad Nacional de Colombia}, Bogotá
    \end{twocolentry}

    \setlength{\parskip}{0.3cm}
    
	\begin{onecolentry}
        \begin{highlights}
            \item Premio Universidad Nacional Mejor Trabajo de Grado Facultad de Biología 2023
            \item Premio Nacional Mejores Resultados Académicos Examen Saber Pro 2023
        \end{highlights}
    \end{onecolentry}
    
	\vspace{-0.8 cm}
	\textcolor{forestgreen}{\section{Experiencia Profesional}}
	
	\begin{twocolentry}{\textcolor{forestgreen}{
			Mayo 2023 -- Diciembre 2024
		}}
		\textcolor{olivegreen}{\textbf{Asistente de Profesorado}}, \textbf{Universidad de Maryland}, College Park, MD, USA. \\
		\textit{Laboratorio de Dr. Juan Angueyra, Ph.D.}
	\end{twocolentry}
	
	\begin{onecolentry}
        \begin{highlights}
		      \item Exploración de la diversidad de subtipos de células horizontales en múltiples especies utilizando marcaje neuronal y microscopía confocal en serpientes, geckos, peces y gerbos.
		      \item Caracterizacion de los efectos de la modulación de la hormona tiroidea en los subtipos de fotorreceptores de \textit{Astatotilapia burtoni}.
		      \item Caracterizacion de factores de transcripción, después de mutagénesis, para evaluar la conectividad sináptica de diferentes subtipos de células de la retina.
		      \item Construcción de equipos de electrofisiología y estimuladores para  registro de electroretinogramas.
        \end{highlights}
	\end{onecolentry}
	
	\vspace{0.2 cm}
	
	\begin{twocolentry}{\textcolor{forestgreen}{
			Enero 2024 -- Mayo 2024
		}}
		\textcolor{olivegreen}{\textbf{Asistente de Docencia}}, \textbf{Universidad de Maryland}, \\
		\textit {Laboratorio de Neurofisiología 405, College Park, USA.\\ Profesora a cargo: Dra. Hillary Bierman, Ph.D.}
		
	\end{twocolentry}
	
	\begin{onecolentry}
		\begin{highlights}
			\item Supervisé de manera independiente un grupo de 20 estudiantes durante sesiones semanales de laboratorio de 4 horas, coordinando y dirigiendo actividades centradas en registros electrofisiológicos extracelulares e intracelulares.
			\item Asesoré a los estudiantes en el diseño y ejecución de sus propios experimentos científicos. Utilizamos diversos modelos experimentales, incluyendo cangrejos de río, grillos y gusanos.
			\item Orienté la presentación de resultados científicos mediante articulos y pósteres. 
		\end{highlights}
	\end{onecolentry}
	
	\vspace{0.2 cm}
	
	\begin{twocolentry}{\textcolor{forestgreen}{Enero 2017 -- Diciembre 2018}
		}
		\textcolor{olivegreen}{\textbf{Profesor voluntario}},\textbf{ Colegio 20 de julio},\\ Bogotá, DC, Colombia
	\end{twocolentry}
	
	\begin{onecolentry}
		\begin{highlights}
			
			\item Brindamos mentoría y acompañamiento a estudiantes provenientes de comunidades con bajos recursos, con una intensidad de 2 horas a la semana, para contribuir al fortalecimiento académico y motivacional de los estudiantes, mejorando sus oportunidades de ingreso a la Universidad Nacional de Colombia.
		\end{highlights}
    
	\end{onecolentry}
	
	\textcolor{forestgreen}{\section{Publicaciones}}
	
	Leon, U., Forero, J. (2025). 
	\textbf{El Observatorio Solar de Bacatá: Nuevas perspectivas arqueoastronómicas en el centro histórico de Bacatá}. 
	eSPECTRA. Revista del Observatorio Nacional de Colombia (En revisión).
	
	
	Bhattarai, J.P., Etyemez, S., Jaaro-Peled, H., Janke, E., León Tolosa, U.D., Kamiya, A., Gottfried, J.A., Sawa, A., Ma, M. (2021).
	\textbf{Modulación olfativa del circuito de la corteza prefrontal medial: implicaciones para la cognición social}.  \href{https://doi.org/10.1016/j.semcdb.2021.03.022}{Seminars in Cell and Developmental Biology}
	
	\vspace{-0.3 cm}
	\textcolor{forestgreen}{\section{Experiencia en Investigación}}
	\begin{twocolentry}{\textcolor{forestgreen}{
			Mayo 2021 -- Diciembre 2022
		}}
		\textcolor{olivegreen}{\textbf{Asistente de Investigación}},\\\textbf{Universidad Nacional de Colombia, Bogotá, D.C.} \\
		\textit{Grupo de Investigación en Neurofisiología} \\
		\textit{Supervisor: Enrico Nasi, PhD}
	\end{twocolentry}
	
	\begin{onecolentry}
		\begin{highlights}
			\item Establecimiento de un protocolo para aislar fotoreceptores funcionales en planaria (\textit{Schmidtea mediterranea}) para explorar la vía de fototransducción mediante mediciones eléctricas.
			\item Marcaje de diferentes elementos de la cascada de fototransducción usando inmunohistoquímica en ocelos de planaria.
			\item Responsable del mantenimiento de planarias  y micro disección de ocelos.
		\end{highlights}
	\end{onecolentry}
	
	\vspace{0.3 cm}
	
	\begin{twocolentry}{\textcolor{forestgreen}{
			Verano 2020}
		}
		\textcolor{olivegreen}{\textbf{Pasantia de Investigacion}}, \\\textbf{University of Pennsylvania, Filadelfia, PA, USA} \\
		\textit{Programa de intercambio de Verano para Pregrado (SUIP)} \\
		\textit{Supervisor: Minghong Ma, PhD} 
	\end{twocolentry}
	
	\begin{onecolentry}
		\begin{highlights}
			\item Análisis de videos e imágenes utilizando inteligencia artificial, específicamente la red neuronal “DeepLabCut”, para estudiar cambios conductuales tras manipulación optogenética de las neuronas de las islas de Calleja en el cuerpo estriado.
			\item Coautor de un artículo sobre las conexiones en red entre áreas olfativas y no olfativas, como el bulbo olfatorio y la corteza prefrontal.
		\end{highlights}
	\end{onecolentry}
	
	\vspace{0.3 cm}
	
	\begin{twocolentry}{\textcolor{forestgreen}{
			Enero 2019 -- Diciembre 2020
		}}
		\textcolor{olivegreen}{\textbf{Investigador en Formación}}, \\\textbf{Universidad Nacional de Colombia, Bogotá, D.C.}, \\
		\textit{Grupo de Investigación en Neurofisiología} \\
		\textit{Supervisor: María Del Pilar Gómez, MD, PhD }
	\end{twocolentry}
	
	\begin{onecolentry}
		\begin{highlights}
			\item Capacitación en fundamentos de registros electrofisiológicos celulares e inmunohistoquímica.
			\item Entrenamiento en disección de ojos de pollo y ensayos proteicos como electroforesis y Western Blot.
			\item Mantenimiento y cultivo celular de neuroblastoma.
		\end{highlights}
	\end{onecolentry}
	
	\vspace{0.3 cm}
	
	\begin{twocolentry}{\textcolor{forestgreen}{
			Enero 2018 -- Diciembre 2018
		}}
		\textcolor{olivegreen}{\textbf{Investigador en Formación}}, \\\textbf{Universidad Nacional de Colombia, Bogotá, D.C.},\\
		\textit{Grupo de Investigación en Proteínas} \\
		\textit{Supervisor: Edgar Reyes, PhD}
	\end{twocolentry}
	
	\begin{onecolentry}
		\begin{highlights}
			\item Trabajo en la purificación de una proteína de \textit{Salvia bogotensis}, utilizando cromatografía de afinidad y exclusión molecular para explorar la actividad insecticida de esta proteína en lepidópteros \textit{Tecia solanivora}.
		\end{highlights}
	\end{onecolentry}
	
	\vspace{-0.5cm}
	
	\textcolor{forestgreen}{\section{Presentaciones}}
	
	\noindent
    \begin{tabular*}{\textwidth}{@{\extracolsep{\fill}} p{0.75\textwidth} >{\raggedleft\arraybackslash}p{0.20\textwidth} }
    \textbf{El Observatorio Solar de Bacatá: Nuevas perspectivas
arqueoastronómicas en el centro histórico de Bogotá.} & 2024 \\
    \end{tabular*}
	\vspace{-0.5cm}
	\begin{onecolentry}
		\mbox{\textbf{León, U.}}, \mbox{Forero, J.} — V Taller de Astronomía de los Andes, Universidad de Nariño, Pasto. 
	\end{onecolentry}
	
	\vspace{0.1 cm}
	
	\noindent
    \begin{tabular*}{\textwidth}{@{\extracolsep{\fill}} p{0.75\textwidth} >{\raggedleft\arraybackslash}p{0.20\textwidth} }
    \textbf{¿Esperando a Godot? Descifrando las reglas de
conectividad sináptica de las células horizontales.} & 2024 \\
    \end{tabular*}
    \vspace{-0.5cm}
	\begin{onecolentry}
		\mbox{\textbf{León, U.}}, \mbox{Angueyra, J.} — Mid Atlantic Zebrafish Meeting (MARZ), Rutgers, NJ.\
	\end{onecolentry}
    
    \vspace{0.2 cm}
    
    \noindent
    \begin{tabular*}{\textwidth}{@{\extracolsep{\fill}} p{0.75\textwidth} >{\raggedleft\arraybackslash}p{0.20\textwidth} }
    \textbf{El ojo de planaria como modelo para estudiar la
    fototransducción.} & 2023 \\
    \end{tabular*}

	\vspace{-0.3cm}
    
	\begin{onecolentry}
		\mbox{Puerto, P.}, \mbox{\textbf{León, U.}}, \mbox{Zuluaga, M.}, \mbox{Gómez, P.} — XIII Seminario Nacional de Neurociencia (COLNE), Cali.
	\end{onecolentry}
	
	\vspace{0.2 cm}

    \noindent
    \begin{tabular*}{\textwidth}{@{\extracolsep{\fill}} p{0.75\textwidth} >{\raggedleft\arraybackslash}p{0.20\textwidth} }
    \textbf{Contribución del sistema olfativo a la cognición social en ratón.} & 2020 \\
    \end{tabular*}
    
    \vspace{-0.3 cm}
    
    \begin{onecolentry}
        \mbox{\textbf{León, U.}}, \mbox{Ma, M.} — Undergraduate Internship Program, University of Pennsylvania, Philadelphia, PA. \
	\end{onecolentry}

	\vspace{-1 cm}
	
	
	\textcolor{forestgreen}{\section{Habilidades técnicas}}
	
	\begin{onecolentry}
		\textcolor{forestgreen}{\textbf{Microscopía:}} Confocal, de hoja de luz, Expansión, en vivo
	\end{onecolentry}
	
	\begin{onecolentry}
		\textcolor{forestgreen}{\textbf{Electrofisiología:}} Patch-clamp, ERG, registros extracelulares, Vizualizacion de calcio (GCaMP6, Fluo-4 AM)
	\end{onecolentry}
	
	
	\begin{onecolentry}
		\textcolor{forestgreen}{\textbf{Técnicas moleculares:}} PCR, CRISPR, Western blot, transfección
	\end{onecolentry}
	
	\begin{onecolentry}
		\textcolor{forestgreen}{\textbf{Tinciones:}} DII, Golgi, inmunocitoquímica
	\end{onecolentry}
	
	\vspace{-1.2cm}
	
	\textcolor{forestgreen}{\section{Software y programación}}
	
	\begin{onecolentry}
		\textcolor{forestgreen}{ \textbf{Software:}}Python, Bash, FIJI/ImageJ, \textmu Manager, Nikon Elements, Zeiss Zen, DeepLabCut, Prusa Slicer
	\end{onecolentry}
	
	\begin{onecolentry}
		\textcolor{forestgreen}{\textbf{Herramientas:}} Linux, Inkscape, OpenSCAD, Git, LaTeX
	\end{onecolentry}
	
	\begin{onecolentry}
		\textcolor{forestgreen}{\textbf{Hardware:}} Impresion 3D, Arduino, Raspberry Pi, soldadura básica de circuitos
	\end{onecolentry}
	\vspace{-1.2cm}
	\textcolor{forestgreen}{\section{Idiomas}}
	
	\begin{twocolentry}{
			Nativo
		}
		\textbf{Español}
	\end{twocolentry}
	
	\begin{twocolentry}{
			C1 Fluido
		}
		\textbf{Ingles}
	\end{twocolentry}
	
	\begin{twocolentry}{
			A2 Básico
		}
		\textbf{Aleman}
	\end{twocolentry}
	
	\vspace{-1cm}
	
	\textcolor{forestgreen}{\section{Referencias}}
	\vspace{-0.3cm}
	\begin{multicols}{2}
		
		\noindent\textcolor{forestgreen}{\textbf{Juan Angueyra, M.D., Ph.D.}} \\
		Profesor Asistente, Departamento de Biología, Universidad de Maryland \\
		\href{mailto:angueyra@umd.edu}{angueyra(a)umd.edu}
		
		
		\noindent\textcolor{forestgreen}{\textbf{María del Pilar Gómez, M.D., Ph.D.}} \\
		Profesora Asociada, Departamento de Biología, Universidad Nacional de Colombia \\
		\href{mailto:mpgomzco@unal.edu.co}{mpgomzco(a)unal.edu.co}
		
		
		
		\noindent\textcolor{forestgreen}{\textbf{Karen Carleton, Ph.D.}} \\
		Profesora Titular, Departamento de Biología, Universidad de Maryland \\
		\href{mailto:kcarleto@umd.edu}{kcarleto(a)umd.edu}
		
		
		\noindent\textcolor{forestgreen}{\textbf{Minghong Ma, Ph.D.}} \\  
		Profesora Titular, Departamento de Neurociencia, Universidad de Pennsylvania \\ \href{mailto:minghong@pennmedicine.upenn.edu.}{minghong(a)pennmedicine.upenn.edu.}
    \end{multicols}
      \begin{twocolentry}{
            
        }
        \textcolor{forestgreen}{\textbf{Enrico Nasi Lignarolo, Ph.D.}} \\
		Profesor titular, Departamento de genética, \\
        Universidad Nacional de Colombia \\
		\href{mailto:enasil@unal.edu.co}{enasil(a)unal.edu.co}
     \end{twocolentry}
		
	
\end{document}
